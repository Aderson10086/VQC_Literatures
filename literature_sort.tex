\documentclass[12pt, oneside]{article}   	% use "amsart" instead of "article" for AMSLaTeX format
\usepackage{geometry}       
\usepackage{hyperref}         		% See geometry.pdf to learn the layout options. There are lots.
\geometry{letterpaper}                   		% ... or a4paper or a5paper or ... 
%\geometry{landscape}                		% Activate for rotated page geometry
\usepackage[parfill]{parskip}    		% Activate to begin paragraphs with an empty line rather than an indent
\usepackage{graphicx}				% Use pdf, png, jpg, or eps§ with pdflatex; use eps in DVI mode
								% TeX will automatically convert eps --> pdf in pdflatex		
\usepackage{amssymb}

%SetFonts

%SetFonts


\title{Quantum Machine Learning Research Topic and Corresponding Literatures}
\author{\LaTeX--yhu}
\date{}							% Activate to display a given date or no date

\begin{document}
\maketitle
\section{Topic 1: Design Variational Quantum Circuit (VQC) Ansatz by AI, eg. Reinforcement Learning}
\begin{itemize}
\item[1. ] A quantum information theoretical analysis of reinforcement	 learning-assisted quantum architecture search \url{https://arxiv.org/pdf/2404.06174.pdf}
\item[2. ] Bayesian Parameterized Quantum Circuit Optimization \url{https://arxiv.org/pdf/2404.11253.pdf}
\item[3. ] Quantum Architecture Search with Unsupervised Representing Learning \url{https://arxiv.org/pdf/2401.11576.pdf}
\item[4. ] Optimizing ZX-Diagrams with Deep Reinforcement Learning \url{https://arxiv.org/pdf/2311.18588} 
\item[5. ] Light-cone feature selection for quantum machine learning \url{https://arxiv.org/pdf/2403.18733} 
\item[6. ] Light Cone Cancellation for VQE Ansatz \url{https://arxiv.org/pdf/2404.19497} 
\end{itemize}



\section{Topic 2: Design VQC with some features, eg. Symmetry, Topology}
\begin{itemize}
\item[1. ] Here comes the $SU(N)$: multivariate quantum gates and gradients \url{https://quantum-journal.org/papers/q-2024-03-07-1275/}
\item [2. ]  Geodesic Algorithm for Unitary Gate Design with Time-independent Hamiltonians \url{https://arxiv.org/pdf/2401.05973.pdf}
\item [3. ] Tensorized Pauli decompsition algorithm \url{https://arxiv.org/pdf/2310.13421}
\end{itemize}

\section{Topic 3: Simulate VQC or in real device, eg. State vector, Tensor}
\begin{itemize}
\item[1.] Hybrid tree tensor networks for quantum simulation \url{https://arxiv.org/pdf/2404.05784.pdf}
\end{itemize}


\section{Topic 4: About Reading-out information  from VQC}
\begin{itemize}
	\item [1. ] Artificial-Intelligence-Driven Shot Reduction in Quantum Measurement \url{https://arxiv.org/pdf/2405.02493}
\end{itemize}

\section{Topic 5: The inductive bias for VQC as machine learning models}
\begin{itemize}
\item[1. ] Inductive Bias for Deep Learning of High-Level Cognition \url{https://arxiv.org/pdf/2011.15091.pdf}
\item[2. ] Contextually and inductive bias in quantum machine learning \url{https://arxiv.org/pdf/2302.01365.pdf}
\item[3. ] The inductive bias of quantum kernel \url{https://arxiv.org/pdf/2106.03747.pdf}
\end{itemize}

\section{Topic 6: Optimize method, eg. Gradient descent, natural gradient, gradient-free or with some features}
\begin{itemize}
\item[1. ] Efficient Gradient Estimation of Variational Quantum Circuit with Lie Algebraic Symmetries \url{https://arxiv.org/pdf/2404.05108.pdf}
\item[2. ] Optimizing Variational Quantum Algorithms with qBang: Efficient Interweaving metric and Momentum to Navigate Flat Energy Landscapes \url{https://quantum-journal.org/papers/q-2024-04-09-1313/}
\item[3. ] Quantum conjugate gradient method using the positive-side quantum eigenvalue transformation \url{https://arxiv.org/pdf/2404.02713.pdf}
\item [4. ] A Novel Noise-Aware Classical Optimizer for Variational Quantum Algorithms \url{https://arxiv.org/pdf/2401.10121.pdf}
\item[5. ] Variational Quantum Simulation: A case study for understanding warm starts\url{https://arxiv.org/pdf/2404.10044.pdf}
\item[6. ] Guided-SPSA: Simultaneous Perturbation Stochastic Approximation assisted by the Parameter shift rule \url{https://arxiv.org/pdf/2404.15751.pdf}
\item[7. ] Improving Gradient Methods via Coordinate Transformations: Applications to Quantum Machine Learning\url{https://arxiv.org/pdf/2304.06768}
\item[8. ] Better Optimization of VQE by Combining the Unitary Block Optimization Scheme with Classical Post-Processing \url{https://arxiv.org/pdf/2404.19027} 
\item[9. ] Quantum Global Minimum Finder Based on Quantum Variational Search \url{https://arxiv.org/pdf/2405.00450} 
\item[10. ] Hybrid Quantum-Classical Scheduling for Accelerating NN Training with Newton's Gradient Descent \url{https://arxiv.org/pdf/2405.00252}
\item[11. ] Training robust and generalizable quantum models \url{https://arxiv.org/pdf/2311.11871}
\item[12. ] Quantum-Informed Recursive Optimization Algorithms \url{https://journals.aps.org/prxquantum/pdf/10.1103/PRXQuantum.5.020327}
\end{itemize}
\section{Topic 7: Mitigation Noise and Barren Plateau}
\begin{itemize}
\item[1. ] Can Error Mitigation Improve Trainability of Noisy Variational Quantum Algorithms? \url{https://quantum-journal.org/papers/q-2024-03-14-1287/}
\item [2. ] Exploiting many-body localization for scable quantum simulation \url{https://arxiv.org/pdf/2404.17560}
\item[3. ] A Review of Barren Plateaus in Variational Quantum Computing \url{https://arxiv.org/pdf/2405.00781}
\item[4. ] Improving Trainability of VQC via Regularization Strategies \url{https://arxiv.org/pdf/2405.01606}
\end{itemize}
\section{Topic 8: Scale Quantum Machine Learning, the most literature aimed at small applications or problems}
\begin{itemize}
\item[1. ] Towards provably efficient quantum algorithms for large-scale machine-learning models \url{https://www.nature.com/articles/s41467-023-43957-x#Sec7}
\item[2. ] Quantum machine learning of large datasets using randomized measurements \url{https://arxiv.org/pdf/2108.01039.pdf}
\item[3. ] QNLP \url{https://arxiv.org/pdf/2403.19758}
\item[5. ] A quantum neural network framework for scalable quantum circuit approximation of unitary matrices \url{https://arxiv.org/pdf/2405.00012}
\end{itemize}
\section{Topic 9: What's the theory guarantee behind if QML better than Classical models}
\begin{itemize}
\item[1. ]Better than classical? The subtle art of benchmarking	quantum machine learning models \url{https://arxiv.org/pdf/2403.07059.pdf}
\end{itemize}
\section{Topic 10: What's kind of problem set can be solved effectively  by QML or VQC}
\begin{itemize}
	\item [1.] What makes data suitable for a locally connectd neural networks ? A necessary and sufficient conditions based on quantum entanglement \url{https://arxiv.org/pdf/2303.11249.pdf}
	\par Analysising the classical model with quantum method by tensor networks.
	\item[2. ] Quantum Sovlable Nonlinear Differential Equations \url{https://arxiv.org/pdf/2305.00653.pdf}
\end{itemize}
\section{Topic 11: Some hot applications, eg. Quantum Generative model, Quantum  Reinforcement Learning}
\begin{itemize}
\item[1. ] VQC-based Reinforcement Learning with Data Reuploading: Performence and Trainability  \url{https://arxiv.org/pdf/2401.11555.pdf}
\item [2. ] On  Quantum Natural Policy Gradients \url{https://arxiv.org/pdf/2401.08307.pdf}
\item [3. ] Variational Quantum Algorithms for Semidefinite Programming \url{https://arxiv.org/pdf/2112.08859}
\item[4. ] Tensor Networks Based quantum Optimize Algorithm \url{https://arxiv.org/pdf/2404.15048}
\item[5 .] Guardians of the Quantum Gan\url{https://arxiv.org/pdf/2404.16156}
\item[6. ] Quantum Speedsup in Regret Analysis of Infinite Horizon Average-Reward Markov Decision Processes \url{https://arxiv.org/pdf/2310.11684} 
\end{itemize}
\section{Topic 12: How to explain QML}
\begin{itemize}
\item[1. ] Understanding quantum machine learning also requires rethinking generalization \url{https://doi.org/10.1038/s41467-024-45882-z}
\item[2. ] Quantum-inspired activation functions in the convolutional neural networks \url{https://arxiv.org/pdf/2404.05901.pdf}
\item[3. ] Bounds and guarantees for learning and entanglement \url{https://arxiv.org/pdf/2404.07277.pdf}
\item[4. ] On the interpretability on Quantum Machine Learning \url{https://arxiv.org/pdf/2308.11098.pdf} 
\par Explainable Quantum Machine Learning and Explainable AI, the references of "on the interpretability on Quantum Machine Learning" are worth reviewing.
\item[5 .] Learning Quantum Processes with Quantum Statistical Query \url{https://arxiv.org/pdf/2310.02075}
\item[6. ] Revealing the working mechanism of quantum neural networks by mutual information \url{https://arxiv.org/pdf/2404.19312}
\item[7. ] Analyzing variational quantum landscapes with information content \url{https://www.nature.com/articles/s41467-023-43957-x#Sec7}
\item[8. ] Theoretical guarantees for permutation-equivariant quantum neural networks \url{https://www.nature.com/articles/s41534-024-00804-1}
\item[9. ] Experimental verification of the quantum nature of a neural network \url{https://arxiv.org/pdf/2209.07577}
\end{itemize}

\section{Topic 13: How to encode the classical data into VQC effectively}
\begin{itemize}
\item[1. ] Efficient quantum amplitude encoding of polynomial functions \url{https://quantum-journal.org/papers/q-2024-03-21-1297/}
\item[2.] Optimal Universal Quantum Encoding for statistical inference \url{https://arxiv.org/pdf/2404.08172.pdf}
\item[3.] Let Quantum  Neural Networks Choose Their Own Frequencies \url{https://arxiv.org/pdf/2309.03279.pdf}
\item [4. ] Approximating Korobov Functions via quantum Circuit \url{https://arxiv.org/pdf/2404.14570}
\item [5. ] A quantum compiler design method by using linear combinations of permutations \url{https://arxiv.org/pdf/2404.18226}
\item [6. ] Qubit encoding for a mixture of localized functions \url{https://arxiv.org/pdf/2404.18529}
\item [7.] Circuit Complexity of Quantum Access models for encoding classical data \url{https://arxiv.org/pdf/2311.11365}
\item [8. ] Understanding the effects of data encoding on quantum-classical convolutional neural networks \url{https://arxiv.org/pdf/2405.03027}
\end{itemize}
\section{Topic 14: Continuous Problem in Quantum World}
\begin{itemize}
\item[1. ] Quantum Kernel Method with Continuous Variable \url{https://arxiv.org/pdf/2401.05647.pdf}
\end{itemize}
\section*{Some New}
\begin{itemize}
	\item [1. ] Fermionic Machine Learning \url{https://arxiv.org/pdf/2404.19032}
\end{itemize}
\end{document}  